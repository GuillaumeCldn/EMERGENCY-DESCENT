\documentclass[12pt,oneside]{scrreprt}

% ---------------------------------------------------------
% PACKAGES
% ---------------------------------------------------------
\usepackage[utf8]{inputenc}
\usepackage[T1]{fontenc}
\usepackage[english]{babel}
\usepackage{graphicx}
\usepackage{subcaption}
\usepackage{amsmath}
\usepackage{siunitx}
\usepackage{booktabs}
\usepackage[hidelinks]{hyperref}
\usepackage{lmodern}
\usepackage{microtype}
\usepackage{float}
\usepackage{fmtcount}
\usepackage{tikz}
\usetikzlibrary{arrows.meta, matrix, decorations.pathreplacing, calc}
\usepackage{wrapfig}
%\usetikzlibrary{graphs, graphdrawing}
%\usegdlibrary{circular}

% Better diff symbol
\newcommand{\diff}{\ensuremath{\operatorname{d}\!}}

% ---------------------------------------------------------
% MISE EN PAGE
% ---------------------------------------------------------
\usepackage[margin=2.5cm]{geometry}

\setlength{\parskip}{0.85em}
\setlength{\parindent}{1.2em}
\linespread{1.15}

% ---------------------------------------------------------
% TITRES (méthode KOMA-Script)
% ---------------------------------------------------------
\RedeclareSectionCommand[
beforeskip=1.8em plus 0.2em minus 0.2em,
afterskip=1em,
font=\normalfont\huge\bfseries]{chapter}

\RedeclareSectionCommand[
beforeskip=1.4em plus 0.2em minus 0.2em,
afterskip=0.8em,
font=\normalfont\Large\bfseries]{section}

\RedeclareSectionCommand[
beforeskip=1em plus 0.2em minus 0.2em,
afterskip=0.5em,
font=\normalfont\large\bfseries]{subsection}

% ---------------------------------------------------------
% TITRE DU RAPPORT
% ---------------------------------------------------------
\title{\Huge \bfseries Functional Requirement Document}

\author{%
	\Large Luc Flohr \\
	\texttt{luc.flohr@alumni.enac.fr}\\[1.2em]
	\Large Balzs Palotas \\
	\texttt{balzs.palotas@alumni.enac.fr}\\[1.2em]
	\Large Guillaume Claudon \\
	\texttt{guillaume.claudon@alumni.enac.fr}\\[1.2em]
	\large École Nationale de l’Aviation Civile (ENAC) %
}

\date{Mars 2026}



% ---------------------------------------------------------

\publishers{
	\includegraphics[width=4cm]{logo_ENAC.png}
}

\begin{document}
% ---------------------------------------------------------

\maketitle


% ===============================================================
\chapter{Executive summary}
% ===============================================================

This document is the functional specification of the EMERGENCY DESCENT function. It provides “aircraft-level” requirements that will be the main input for systems specifications.

The purpose of the EMERGENCY DESCENT function is to have the aircraft descend to a safe altitude in the case of loss of cabin pressurisation.

Some requirements need to be consolidated, in particular those involving response time (refer to §5.1)

\tableofcontents
\nopagebreak
\listoffigures
\clearpage

% ===============================================================
\chapter{Record of revisions}
% ===============================================================
% ===============================================================
\chapter{General}
% ===============================================================
% ===============================================================
\chapter{Function description}
% ===============================================================
% ===============================================================
\chapter{Operational scenarios}
% ===============================================================
% ===============================================================
\chapter{Requirements}
% ===============================================================
% ===============================================================
\chapter{Open items}
% ===============================================================
% ===============================================================
\chapter{Appendices}
% ===============================================================

\end{document}
